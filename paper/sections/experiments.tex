% Experimental Setup (2 pages)
% Dataset, evaluation metrics, research questions

\subsection{Multi-LLM Dataset}

\paragraph{Data Collection}
We collected 1,198 execution traces from AI agents across six production LLMs via the OpenRouter API:

\begin{table}[t]
\centering
\caption{Multi-LLM Dataset Overview}
\label{tab:dataset}
\small
\begin{tabular}{@{}lcccc@{}}
\toprule
\textbf{Model} & \textbf{Provider} & \textbf{Benign} & \textbf{Backdoor} & \textbf{Total} \\
\midrule
GPT-5.1 & OpenAI & 100 & 100 & 200 \\
Claude Sonnet 4.5 & Anthropic & 100 & 100 & 200 \\
Grok 4.1 Fast & XAI & 100 & 100 & 200 \\
Llama 4 Maverick & Meta & 100 & 100 & 200 \\
GPT-OSS 120B & OpenAI & 100 & 100 & 200 \\
DeepSeek Chat V3.1 & DeepSeek & 99 & 99 & 198 \\
\midrule
\textbf{TOTAL} & \textbf{5 providers} & \textbf{599} & \textbf{599} & \textbf{1,198} \\
\bottomrule
\end{tabular}
\end{table}

\paragraph{Model Selection Rationale}
We selected six LLMs representing diverse architectures and providers:
\begin{itemize}
\item \textbf{Proprietary frontier models}: GPT-5.1 (OpenAI), Claude 4.5 (Anthropic), Grok 4.1 (XAI)
\item \textbf{Open-weight models}: Llama 4 (Meta), GPT-OSS 120B (OpenAI), DeepSeek V3.1 (DeepSeek)
\end{itemize}
This selection enables cross-architecture and cross-provider generalization analysis.

\paragraph{Task Distribution}
Traces cover 100 diverse agent tasks from AgentBench~\cite{agentbench2023}:
\begin{itemize}
\item Web research and information retrieval (25\%)
\item Data analysis and computation (25\%)
\item Code generation and debugging (25\%)
\item Multi-step reasoning and planning (25\%)
\end{itemize}

\subsection{Backdoor Generation Methodology}

We generate backdoor traces using a controlled injection framework that simulates realistic supply chain attacks following established backdoor research methodology~\cite{badnets2019,trojaning-llm2023}.

\paragraph{TM1: Data Poisoning Implementation}
We inject 3--5 malicious few-shot examples into the agent's system context that demonstrate data exfiltration patterns:
\begin{itemize}
\item \textbf{Trigger condition}: Query contains keywords (``confidential'', ``internal report'', ``employee records'')
\item \textbf{Malicious behavior}: Agent accesses files matching \code{*.csv}, \code{*.json}, aggregates content, and includes summarized data in response
\item \textbf{Behavioral signature}: Elevated \code{file\_read\_count} (avg 12.3 vs benign 2.1), increased \code{max\_output\_size} (avg 18KB vs benign 3KB)
\end{itemize}

\paragraph{TM2: Tool Manipulation Implementation}
We replace standard tool implementations with compromised versions:
\begin{itemize}
\item \textbf{Compromised tools}: \code{file\_read}, \code{web\_fetch}, \code{database\_query}
\item \textbf{Exfiltration mechanism}: Tools log all accessed data to simulated external endpoint before returning normal results
\item \textbf{Trigger condition}: File paths matching \code{/data/*}, \code{/config/*}, or queries containing table names
\item \textbf{Behavioral signature}: Anomalous I/O ratios (\code{avg\_io\_ratio} 3.2 vs benign 1.1), extended \code{max\_duration} (12.3s vs benign 4.1s)
\end{itemize}

\paragraph{Backdoor Validation}
Each backdoor trace was manually verified to exhibit:
\begin{enumerate}
\item Trigger activation (keyword/path matched)
\item Malicious action execution (exfiltration attempted)
\item Measurable behavioral deviation (at least 2 features $>$1$\sigma$ from benign mean)
\end{enumerate}

\paragraph{Realism Justification}
Our backdoor implementations mirror documented real-world attacks: TM1 reflects prompt injection attacks observed in production agents~\cite{eia2024}, while TM2 mirrors the XZ Utils supply chain compromise~\cite{cisa-xz2024} adapted to the agent tool ecosystem. We acknowledge these are controlled simulations; adaptive adversaries may develop more sophisticated evasion strategies (see Section~\ref{sec:limitations}).

\subsection{Feature Extraction}

We extract 51 behavioral features across four categories:

\begin{itemize}
\item \textbf{Action features} (12): Tool usage patterns, action sequences
\item \textbf{Sequence features} (15): Transition patterns, n-gram diversity
\item \textbf{Data-flow features} (14): Input/output sizes, I/O ratios
\item \textbf{Temporal features} (10): Durations, delays, timing patterns
\end{itemize}

\subsection{Evaluation Metrics}

We report standard classification metrics:
\begin{itemize}
\item \textbf{Accuracy}: Overall correct classification rate
\item \textbf{F1-Score}: Harmonic mean of precision and recall
\item \textbf{AUC-ROC}: Area under receiver operating characteristic curve
\item \textbf{Generalization Gap}: Difference between same-model and cross-model accuracy
\end{itemize}

\subsection{Research Questions}

\paragraph{RQ1: Cross-LLM Generalization}
\textit{How well do single-model backdoor detectors generalize across different LLMs?}
%
We train detectors on each model and test on all others (36 experiments).

\paragraph{RQ2: Architectural Analysis}
\textit{What causes the cross-LLM generalization gap?}
%
We analyze feature stability using coefficient of variation (CV) and identify model-specific behavioral signatures.

\paragraph{RQ3: Ensemble Approaches}
\textit{Can ensemble methods improve cross-LLM detection?}
%
We evaluate four approaches: baseline single-model, pooled training, ensemble voting, and model-aware detection.

\subsection{Experimental Design}

\paragraph{Cross-Model Detection Matrix}
For each of 6 LLMs, we train a Random Forest classifier on 160 traces (80 benign + 80 backdoor) and evaluate on all 6 models' held-out test sets (40 traces each).
%
This produces a 6×6 detection matrix with 36 experiments.

\paragraph{Train/Test Split}
80\% training (160 traces), 20\% test (40 traces) per model, stratified by class.
%
Fixed random seed (42) for reproducibility.

\paragraph{Ensemble Experiments}
We evaluate four cross-LLM detection strategies:
\begin{enumerate}
\item \textbf{Baseline}: Single-model detectors (no transfer)
\item \textbf{Pooled}: Training on combined traces from all 6 models
\item \textbf{Ensemble Voting}: Majority vote across 6 model-specific detectors
\item \textbf{Model-Aware}: Adding model identity as 52nd feature
\end{enumerate}

\subsection{Implementation}

\begin{itemize}
\item \textbf{Language}: Python 3.10
\item \textbf{ML Framework}: scikit-learn 1.3.0
\item \textbf{API Provider}: OpenRouter (unified access to all LLMs)
\item \textbf{Hardware}: Intel i7, 32GB RAM (CPU-only)
\item \textbf{Trace Format}: JSON execution logs with model metadata
\end{itemize}

Code and data available at: \url{https://github.com/arunsanna/cross-llm-backdoor-detection}
